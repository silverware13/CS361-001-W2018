\documentclass[12pt]{article}
%\usepackage{times}
\usepackage{cite}
\usepackage{url}
\usepackage{graphicx}
\setlength{\parskip}{1em}
\setlength{\parindent}{0em}
%this is a comment
\title{CS 361 Software Requirements and Planning : \\ Digital Ghostwriter}
\author{Thomas Hollenberg (hollenbt), Zachary Thomas (thomasza),\\ Amar Raad (raadv), Jared Tence (tencej),\\ Zech DeCleene (decleenz)}
\date{January 23rd, 2018}

\begin{document}
\maketitle
\tableofcontents

\newpage

\section{Project Description}

\subsection{Approach (i.e., proposed system)}

Digital Ghostwriter is a Linux desktop application that brings the advantages of ghostwriting to the masses through the power of deep learning.

The application uses an existing open-source tool that uses multi-layer recurrent neural networks to generate content in the style of the training material. The main value added is in generating training material based on user specifications, as well as post-processing to fix syntactical errors in the output, although the underlying model may be modified if time allows (given the complexity and steep learning curve of deep learning concepts).

This application allows users to write books, poetry, or songs that they may never have had the time or motivation to create. It also enables authors with writer's block to explore ways in which they themselves could advance their own stories.

\subsection{Previous approaches}

Most previous approaches of book writing AI rely on a user being shown a selection of the next most probable word and then having the user select the word that they find most appealing. While this approach does allow control over the end result it also is very time intensive.\cite{JapaneseAIWritesaNovel}

The application being presented here instead has the user set some stylistic options at the beginning such as genre and word count. The language model is has already been trained based on a compilation of literary works pulled from the Internet that match the style specifications. Completed literature can be generated by specifying genre and word count with no further user input, thus making it very time efficient.

An acceptable limitation of our design is that the depth of control allowed by pre-generation options will not compete with that of having control over every word in the book.


\subsection{Resource interaction diagram}

\begin{figure}[h]
  \centering
    \includegraphics[scale=0.5]{diagram.eps}
    \caption{Software diagram.}
    \label{fig:Software diagram.}
\end{figure}

\subsection{Programming language and tools:}

The application will be built around Sung Kim's word-rnn-tensorflow tool, which uses Multi-layer Recurrent Neural Networks (LSTM, RNN) for word-level language models in Google's Python-based Tensorflow machine learning framework.\cite{WordRNNTensorflow}

Word-rnn-tensorflow allows us to generate predictive writing and is a perfect fit for our project. For this reason, the application will be written in Python (specifically pyGUI so as to allow us to make a GUI for easier user interaction) and will require that Tensorflow is installed to be run. Digital Ghostwriter will be written for Linux operating systems. The application will require Internet access to generate training material. HTTP requests will possibly be made with Python's urllib.request package. Ghostwritten content will be output as a plain text document.

\subsection{Features (Functional Requirements)}
\subsubsection{Major features}
\begin{enumerate}
\item Graphical user interface
\begin{itemize}
\item Button to select from default genres
\item Button to select from saved user-defined genres
\item Button to create a new user-defined genre
\item Button to delete an existing user-defined genre
\item Text field to select word count
\item Button to generate output based on word count and selected genre
\end{itemize}

\item Default genres - The user will be able to select from the following (pre-trained) default genres:
\begin{itemize}
\item Comedy
\item Drama
\item Horror
\item Science Fiction
\item Mystery
\end{itemize}

\item User-defined genres - The user will be able to define their own genres by inputting their own training text on which the word-rnn-tensorflow model will be trained. These user-defined genres will be saved for future use, but can be deleted at will.
\item Predictive text generation with word-rnn-tensorflow
\item Generated text syntax checking - Post-processing to clean up syntax errors such as capitalizing the beginning of sentences and closing all opening quotation marks
\end{enumerate}

\subsubsection{Stretch goals}
\begin{enumerate}
\item Custom web search for new genre definition
\item Generated text semantic checking
\end{enumerate}

\subsection{Non-functional Requirements}
\begin{enumerate}
\item The user interface must be self-explanatory (see User Documentation section below).
\item Each default genre must contain at least three different authors.
\item There should be no obvious punctuation errors in the output text.
\end{enumerate}


\subsection{User Documentation}
All GUI elements (buttons, text fields, etc.) will have hover text that gives a one or two sentence description of the requirements for and consequences of interacting with that element. For example, the word count text field will specify that it must contain a positive integer indicating the word count of the ghostwritten document. Additionally, a user manual will be provided with the application that provides step by step instructions for each feature.

\newpage

\section{Use Cases}

The following use cases cover the primary three user interactions with the system. The main functionality is the ability to select a genre and word count and have the system create a story for the user in a text file which is what is shown in case 1.

The system selects words based on authors' works that are added to the system, if the user is not happy with the selection that the program ships with they have the option to add documents to the system to create a new genre, case 2 covers this process.

If a user is curious about the authors that are being referenced by the system to create specific genres the user has the option to get a list of all authors being attributed to a given genre. Use case 3 shows this process.

{\Large\textbf{Use case one.}\par}

\textbf{Use case name:} Create short story.

\textbf{Goal:} Generate a short story matching the users desired settings. 

\textbf{Actor:} User who wants to generate a short story.

\textbf{Preconditions:} User has a computer and the Digital Ghostwriter software.

\textbf{Postconditions:} System has successfully generated a text file containing a short story.

\textbf{Flow of events:}
\begin{itemize}
\item User selects generate text option.
\item User selects a name for their text.
\item User selects their preferred genre. 
\item User selects a word count.
\item System generates a text file with the given name and the contents matching the selected genre and word count.
\end{itemize}

{\Large\textbf{Use case two.}\par}

\textbf{Use case name:} Create new genre

\textbf{Goal:} Generate a new genre using text documents submitted by the user. 

\textbf{Actor:} User who wants a customized genre.

\textbf{Preconditions:} User has a computer and the Digital Ghostwriter software.

\textbf{Postconditions:} System has successfully generated a genre that will use those text documents in it's protocol for generating words.

\textbf{Flow of events:}


\begin{itemize}
\item User selects create genre option.
\item User submits documents that the genre is based off of.
\item System generates a Genre using submitted documents as the format
\end{itemize}

{\Large\textbf{Use case three.}\par}

\textbf{Use case name:} Show authors.

\textbf{Goal:} Allows users to see what authors works are being used in a specific genre.

\textbf{Actor:} User who wants to see what authors are in a genre.

\textbf{Preconditions:} User has a computer and the Digital Ghostwriter software.

\textbf{Postconditions:} System shows a list of authors in a genre to the user.

\textbf{Flow of events:}

\begin{itemize}
\item User selects show authors option.
\item User selects a genre.
\item System shows a list of all the authors who have works added to that genre.
\end{itemize}


{\Large\textbf{Main error scenarios.}\par}

\begin{enumerate}
\item \textbf{Scenario:} User enters a huge word count for their story.

\textbf{Solution:} System will have a range of valid word counts to prevent excessively large files.
\item \textbf{Scenario:} User enters an invalid genre.

\textbf{Solution:} System will give user a prompt to reenter genre if an invalid one is selected.

\item \textbf{Scenario:} User enters a name for their story that is already taken by a different text file.

\textbf{Solution:} System will give user a prompt to reenter name if a text file with that name already exists.

\item \textbf{Scenario:} User attempts to add a document of the wrong file type with the new genre option.

\textbf{Solution:} System will tell the user that the file type is not valid and allow them to select a different file.

\end{enumerate}

\newpage

\section{Planning}
\subsection{Milestones/Tasks}
\subsubsection{Main features}
\begin{itemize}
\item Learn Python basics.
\item Install required software on all team members' machines.
\item Create a user friendly GUI.
  \begin{itemize}
  \item Choose a framework (most likely PyGUI).
  \item Create throwaway prototype.
  \item Create evolutionary prototype.
  \item Integrate functionality button by button.
  \end{itemize}
\item Integrate word-RNN into our project to allow predictive text.
\item Construct and train models for default genres
\item Implement user-defined genres (creation/deletion/editing/renaming).
\item Implement syntax check post-processing.
\begin{itemize}
\item Define syntax rules.
\item Build scanner to enforce them.
\end{itemize}
\end{itemize}

\subsubsection{Stretch Goals}
\begin{itemize}
\item Implement custom web search for user-defined genre training material.
\item Implement semantic check post-processing.
\end{itemize}


\subsection{Schedule/Timeline}

2/4/2018:
All members have Python 3.6, TensorFlow 1.0, PyGUI, and word-rnn-tensorflow downloaded to their computers.
Team creates a throwaway prototype of the GUI.

2/11/2018:
We split off into two sub groups, Python GUI and word-rnn-tensorflow.
The Python GUI team works on creating a basic prototype of the graphic interface.
The other group will begin modifying word-rnn-tensorflow to allow selection of the desired model file and begin constructing the default genres.

2/18/2018:
Continue progress on previous tasks.
If Python GUI team is done creating the basic prototype, they can look into evolving it for multiple genres.
If the other group is done modifying word-rnn-tensflow, they can look into implementing syntax check post-processing and user-defined genres.

2/25/2018:
At this point the GUI should be able to interact with the word-rnn-tensorflow program. complete near final prototype.

3/4/2018 - onwards: 
Refine and complete prototype if time allows add stretch functionality.


\subsection{Project tracking}

To track our progress on this project we will meet weekly, every Sunday at 2pm, to report our progress on our assigned task. This will give us time to discuss what we have done, what we need help on, and what next we need to work on. We can choose to hold these meetings in person or on discord.

\subsection{Risk Management}

\begin{itemize}
\item Risk 1: Implementing word-rnn-tensorflow to read our documents and make the genre for the program so that it produces meaningful words. To avoid this we could look at the types of content we are sending to the word-rnn-tensorflow as well as using the beam search function.
\item Risk 2: Implementing the Python GUI so that it works with the word-rnn-tensorflow program. Our team doesn't have that much experience with Python as we haven't used if before. To avoid this we should spend time learning about the Python GUI and create a tyGUI, Throwaway prototype of the GUI. 
\item Risk 3: The project being incomplete due to a lack of available time. Mid-terms, Finals, homework assignments, and jobs restrict our ability to work on the project and could result in a incomplete project. To avoid this it is important that we divide the project up as evenly as we can and provide support throughout the processes.
\end{itemize}

\section{Meeting Report }

\subsection{This week's progress:}
Created a solid definition of our project and its requirements. Planned a schedule for the team. Worked with team and established long term goals.

\subsection{Goals for next week:}
Making sure all team members have Python3.6, word-rnn-tensorflow, PyGUI on their systems.
Create a throwaway prototype of the GUI.

\subsection{Contribution of team members:}
All team members gave feedback and worked on our planning documentation as well as agreeing on a meeting day (Sunday).

\subsection{Were customers able to meet with the team:}
Yes. Team including customers met at the end of class last Thursday (1/25) and then again while finalizing this document (1/28).

\bibliography{myref}
\bibliographystyle{plain}

\end{document}
