\documentclass[12pt]{article}
%\usepackage{times}
\usepackage{cite}
\usepackage{url}
%this is a comment
\title{CS 361 Vision Statement: \\ DigitalGhostwriter}
\author{Thomas Hollenberg (hollenbt), Zachary Thomas (thomasza)}
\date{January 15th, 2018}



\begin{document}
\maketitle
\tableofcontents


\newpage

\section{Problem Statement}

Writing, whether it be a novel, poem, journal article, speech or song, is difficult and time-consuming. Many writers struggle with writer's block. A writer experiencing writer's block may never get past this personal hurdle and may become frustrated and abandon their work. Others lack the motivation or are simply unable to find time to write due to leading busy lives.\cite{WritingChallenges}

A ghostwriter is a person hired to write literary works, speeches, or other texts that are publicly credited to another person as the author. Hiring a ghostwriter to write memoirs is extremely common for athletes, politicians (including President Trump), and other public figures. Ghostwriters are also employed by publishers to assist a popular author who can no longer keep up with demand for their works.\cite{HardyBoysInvisibleAuthors}

However, hiring a ghostwriter is prohibitively expensive for most people. For a book, the cost typically ranges from \$36,000 to \$100,000, depending on the length and complexity, while hourly rates for shorter content are generally around \$100 per hour.\cite{HiringGhostwriter}

\section{Solution Description}

DigitalGhostwriter (tentative title) is a Linux desktop application that brings the advantages of ghostwriting to the masses through the power of deep learning.

The application uses an existing open-source tool that uses multi-layer recurrent neural networks to generate content in the style of the training material. The main value added is in generating training material based on user specifications, as well as post-processing to fix syntactical errors in the output, although the underlying model may be modified if time allows (given the complexity and steep learning curve of deep learning concepts).

This application allows users to write books, poetry, or songs that they may never have had the time or motivation to create. It also enables authors with writer's block to explore ways in which they themselves could advance their own stories.

\section{Different approaches}

Most previous approaches of book writing AI rely on a user being shown a selection of the next most probable word and then having the user select the word that they find most appealing. While this approach does allow control over the end result it also is very time intensive.\cite{JapaneseAIWritesaNovel}

The application being presented here instead has the user set some stylistic options at the beginning such as format (i.e. song, poem, novel), genre, or author(s). The language model is then trained based on a compilation of literary works pulled from the Internet that match the style specifications. Once trained, completed literature can be generated by specifying a word count with no further user input, thus making it very time efficient.

An acceptable limitation of our design is that the depth of control allowed by pre-generation options will not compete with that of having control over every word in the book.

\section{System Architecture}

The application will be built around Sung Kim's word-rnn-tensorflow tool, which uses Multi-layer Recurrent Neural Networks (LSTM, RNN) for word-level language models in Google's Python-based Tensorflow machine learning framework.\cite{WordRNNTensorflow}

For this reason, the application will most likely be written in Python, and will require that Tensorflow is installed to be run. DigitalGhostwriter will be written for Linux operating systems. The application will require Internet access to generate training material. HTTP requests will possibly be made with Python's urllib.request package. Ghostwritten content will be output as plaintext.

\section{Challenges}

The biggest challenge with this project is having the application write a text that could have believably been written by a person and is not just a mess of incoherent words smashed together. The best way for us to reduce the occurrence of incoherence is to include a post-processing step that fixes syntax, and an (ideally) format-specific processing step that deals with semantics (i.e. less strict for poetry than non-fiction).

A second challenge will be dealing with complex deep learning algorithms. However, this challenge is mitigated by the fact that a suitable open-source tool has already been identified, so nothing must be written from scratch.

Yet another challenge will be scouring the Internet for appropriate training material. One approach might be to find certain websites that contain large databases of poems, song lyrics, free classic literature, etc. and limit our searches to those domains.

\bibliography{vision_statement_ref}
\bibliographystyle{plain}

\end{document}
